\documentclass[final,5p,times,twocolumn,authoryear]{article}

\usepackage{cite}

\begin{document}

\title{Experiment Design}
\author{Jiazhi Zhou}
\maketitle

\section{Overview}

We are designing an user study to study the effects of an interactive ai
agent on how users interact with it, what users prefer and not prefer,
and what drives user's to employ full body interactions.

\section{User study design}

We will place users in front of a camera with a screen, projecting a AI
dance agent. We will instruct the users to try and interact with the
agent first, for a few minutes to warm up, try different things, and see
how they respond. We will then start the timer for around 5 minutes for
the users to engage with the model fully. To avoid distractions, we
should not interact with the users directly, and only respond when they
are disengaged or need some immediately answers to their concerns.

At first, we will employ a full turn-based model for users to interact
with. We will have a obvious turn signal for users to understand whose
turn it is to make a move. And following the user's turn, the AI model
will generate a dance sequence based on the user's dance and to the
music currently playing.

Eventually, we want to add in other modes of turn-taking such as a fluid
turn-taking where there is no need for a explicit signal of whose turn
it is, and users can imply that from motion cues. And a mode where there
is no turn taking and the ai model will continuously generate dance
moves based on the user's dances.

\section{Measurements}

Our goal is to understand what mode of interaction the user prefers, and
why. So a simple Likert Scale will be used for a quick feedback session
after the experiemnt, for users to give scores to the different
interaction modes. After which, a more extensive questionaire should be
provided asking the users on why they preferred what they preferred, and
whether they could tell a difference at all.

Quantitative metrics can also be gathered such as duration of
interaction, user's energy level, infered from their movement, and the
amount of different limbs the users used while trying to interact with
the AI model.

\section{Analysis}

Themes will be analysed within the user's qulitative responnses, to
gather insights into what users perfers, and what can be improved to
prompt more interactions.

The quantitative metrics can be used to cross-reference the user's
subjective answers. And the user's subjective answers can also be used
to judge how good the metrics are in the ability to infer enthusiasm and
overall experience.

Other quantitative metrics, such as interaction duration, and different
limbs used, can be directly analyzed to observe the performance of the
models, in getting the user's to interact with the model in a
full-bodied manner.

\end{document}
\endinput
