\documentclass[final,5p,times,authoryear]{article}
\usepackage{cite}
\begin{document}

\title{Living Document}
\author{Jiazhi Zhou}
\maketitle

\begin{abstract}

This document is used to track progress of the experiments, and literatures read.

\end{abstract}

\section{Introduction}
% a few motivating statements

TODO: write.

\section{Related Work}
% a list of all papers you’ve read, in clusters of similarity, with a few comments on what it was, what was relevant for your project
% some papers will be less relevant, used for context, other papers that are more relevant you can spend more space describing

TODO: write.

\section{Project Context}
% briefly describe goal to support embodied improvisation in VR, for performance, public installation, home use, equipment, etc.
% requirements for ai agent

Goal:

The goal of this project is to explore the capabilities of generative AI models
in supporting various embodied creativity exercises. These include embodied
dance improvisations, performances, public installations and so on.

LumnAI (add ref) is used for public installation. However it is a algorithm and
not a generative deep learning model.

requirements:

For the generative AI to add to the embodied creativity, there are some
requirements that the AI model has to fulfill.

1. Run fast
The AI model has to be able to run quickly. As most machine learning clusters
are not connected to the internet, and would be useless for actively using for a
project like this.

The model has to be able to run on a standalone device, such as a PC or a
laptop. This requires the model to be small enough to load, and run fast enough
so the user feels minimal lag in their input and the models' outuput

2. Small Size
If the model is too large, it would not load on a easily portable device, like a
laptop or a PC, making it impossible to be used.

3. Take in user input
This requirement takes affect in two senses. One is that, the model has to have
the user input as a condition that it generates on. The other is that the model
has to be able to process any user input.

Machine learning model tends to focus on the data that they are trained on, and
if a model only generates well on its training data, it would not be good for a
public installation, as we can not predict user movements.

\section{Procedures}
% what AI agents you’ve explored so far, main training set and characteristics of each
% describe the goal for re-seeding of the ai agent, and technical details of doing so, how you’ve extended the ai agent.
% describe the parameter explorations that you’ve done

What AI Agents we have explored so far.

The existing "dance generation" AIs were reviewed, and filtered based on
availability as well as fit for purpose. The models that were publicly
available, and fit our requirement, of generating future dance based on dance
input, were evaluated. The most fit for purpose model was picked to be the
Bailando++ model. The MNET model was also seen as a viable choice but due to its
complications in training, we have deferred the use of this model for the
current research till a later date. The DanY (Dance With You) model is the most
fit for purpose, but the model size and training complexity made it out of scope
for this current project.

\subsection{Bailando++}

Bailando++ is a model that we explored. This model fits our base requirements of
being small, fast to run, and can be configured to have user's dance input as a
generation condition.

\subsubsection{Model Details}
This model uses a vector quantized variational auto encoder (VQVAE). The use of
this module at the start, to quantize any dance inputs, mean the actual
generation step only uses a latent space representation of the input data. But
this highlights the importance of the quality of the VQVAE. As this mean all
inputs will be quantized by the encoder, and if the encoder fails to encode and
decode input in a believable way, the user input will be skewed, and the output
would not give the user the impression that their input took affect. 

\subsubsection{}


\end{document}